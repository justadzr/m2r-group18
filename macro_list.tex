% \res{}{}: residue
% \ord, ind: operator with the same name
% \w: omega
% \ve: variable epsilon
% \s: sigma
% \D: capital delta
% \Z: integers
% \R: reals
% \F: mathbb F
% \N: natural numbers
% \Q: rationals
% \C: complex
% \A: affine
% \Lam: capital lambda
% \coker: cokernel
% \kp: kappa
% \doubp: double parentheses
% \lbd: \lambda
% \Re, \Im: real, imaginary part
% \ns: mathcal S
% \M: mathcal M
% \To: torus 1
% \Too: torus 2 (for maps between complex tori)
% \mtx: box matrix; probably useless
% \vp: variable phi
% \norm: norm
% \proj, \lcm: operator with the same name
% \leg{}{}: the Legendre symbol
% \sgn, \mult: operator with the same name
% \ft: mathscr F
% \rad, \Spec, \Proj, \MaxSpec, \Gal, \im, \Hom, \height, \id:
% operator with the same name
% \Top, \Sch, \op: for categories
% \comment{}: can be used to make a block comment
% \begin{psmallmatrix}...\end{psmallmatrix}: inline small matrix, round brackets.
% \begin{bsmallmatrix}...\end{bsmallmatrix}: inline small matrix, square brackets.
% Updated May. 28th Yourong Zang
