\documentclass[12pt]{article}
    %%%%%%%%%%%%%%%%%%%%%%%%%%%%%%%%%%%%%%%%%%%%%%%%%%%
    % Please do not add weird packages unless necessary
    %%%%%%%%%%%%%%%%%%%%%%%%%%%%%%%%%%%%%%%%%%%%%%%%%%%
    \usepackage{indentfirst}
    \usepackage{amsmath}
    \usepackage{amssymb}
    \usepackage{amsthm}
    \usepackage[utf8]{inputenc}
    \usepackage{geometry}
    \usepackage{diagbox}
	\usepackage{enumerate}
    \usepackage{siunitx}
    \usepackage{graphicx}
    \usepackage{multirow}
    \usepackage{xcolor}
    \usepackage{tikz-cd}
    \usepackage{mathrsfs}
    \usepackage{bm}
    \usepackage[colorlinks = true,
            linkcolor = blue,
            urlcolor  = blue,
            citecolor = blue,]{hyperref}
    \usepackage{cleveref}    
    
    %%%%%%%%%%%%%%%%%%
    % Tile and Authors
    %%%%%%%%%%%%%%%%%%
    \title{Elliptic Curves}
    \author{Group 18}
    
    %%%%%%%%%%%%%%%%%%%%%%%%%%%%%%%%%%%%%%%%%%%%%%%%%%%
    % Use \begin{theorem}/{lemma}/{corollary} to access
    % Use \begin{proof}...\end{proof}
    %%%%%%%%%%%%%%%%%%%%%%%%%%%%%%%%%%%%%%%%%%%%%%%%%%%
	\newtheorem{theorem}{Theorem}[subsection]
	\newtheorem{lemma}{Lemma}[subsection]
	\newtheorem{corollary}{Corollary}[subsection]
	\theoremstyle{remark}
	\theoremstyle{definition}
	\newtheorem{remark}{Remark}[subsection]
	\newtheorem{example}{Example}[subsection]
	\newtheorem{definition}{Definition}[subsection]
	\newcommand*{\lemmaautorefname}{Lemma}
	\newcommand*{\definitionautorefname}{Definition}
	\newcommand*{\exampleautorefname}{Example}
	\newcommand{\remarkautorefname}{Remark}
	\newcommand*{\corollaryautorefname}{Corollary}
	
    %%%%%%%%%%%%%%%%%%%%%%%%%%%%%%%%%%%%%%%%%%%%%%%%%%%%%%%%%%%%%%%
    %%% List of macros see macro_list.tex
    %%% Please update the list after adding 
    %%% new macros with the date and your name
    %%% Do NOT use \H for the upper half plane. Define \bbH instead
    %%% Do NOT change any used macros
    %%%%%%%%%%%%%%%%%%%%%%%%%%%%%%%%%%%%%%%%%%%%%%%%%%%%%%%%%%%%%%%
    \newcommand{\res}[2]{\underset{#1}{\,\operatorname{Res}\,}#2}
    \newcommand{\ord}[0]{\operatorname{ord}}
    \newcommand{\ind}[0]{\operatorname{ind}}
    \newcommand{\w}[0]{\omega}
    \newcommand{\ve}[0]{\varepsilon}
    \newcommand{\s}[0]{\sigma}
    \newcommand{\D}[0]{\Delta}
    \newcommand{\Z}[0]{\mathbb{Z}}
    \newcommand{\R}[0]{\mathbb{R}}
    \newcommand{\F}[0]{\mathbb{F}}
    \newcommand{\N}[0]{\mathbb{N}}
    \newcommand{\Q}[0]{\mathbb{Q}}
    \newcommand{\C}[0]{\mathbb{C}}
    \newcommand{\A}[0]{\mathbb{A}}
    \newcommand{\Lam}[0]{\Lambda}
    \newcommand{\coker}[0]{\operatorname{coker}}
    \newcommand{\kp}[0]{\kappa}
    \newcommand{\doubp}[1]{\left(\left(#1\right)\right)}
    \newcommand{\lbd}[0]{\lambda}
    \renewcommand{\Re}[0]{\operatorname{Re}}
    \renewcommand{\Im}[0]{\operatorname{Im}}
    \newcommand{\nS}[0]{\mathcal{S}}
	\newcommand{\M}[0]{\mathcal{M}}
    \newcommand{\To}[0]{\mathbb{C}/\Lambda}
    \newcommand{\Too}[0]{\mathbb{C}/\Lambda'}
    \newcommand{\mtx}[4]{\begin{bmatrix}#1 & #2\\ #3 & #4\end{bmatrix}}
    \newcommand{\vp}[0]{\varphi}
    \newcommand{\norm}[1]{\left\lVert#1\right\rVert}
    \newcommand{\proj}[0]{\operatorname{proj}}
    \newcommand{\lcm}[0]{\operatorname{lcm}}
    \newcommand{\leg}[2]{\left(\frac{#1}{#2}\right)}
    \newcommand{\sgn}[0]{\operatorname{sgn}}
    \newcommand{\mult}[0]{\operatorname{mult}}
    \newcommand{\ft}[0]{\mathscr{F}}
    \newcommand{\rad}[0]{\operatorname{rad}}
    \newcommand{\Spec}[0]{\operatorname{Spec}}
	\newcommand{\Proj}[0]{\operatorname{Proj}}
    \newcommand{\MaxSpec}[0]{\operatorname{MaxSpec}}
    \newcommand{\Gal}[0]{\operatorname{Gal}}
    \newcommand{\im}[0]{\operatorname{im}}
    \newcommand{\Hom}[0]{\operatorname{Hom}}
    \newcommand{\height}[0]{\operatorname{height}}
    \newcommand{\id}[0]{\operatorname{id}}
    \newcommand{\comment}[1]{}
    \newcommand{\Top}[0]{\mathsf{Top}}
	\newcommand{\Sch}[0]{\mathsf{Sch}}
    \newcommand{\op}[0]{\mathsf{op}}
    \newenvironment{psmallmatrix}
	{\left(\begin{smallmatrix}}
	{\end{smallmatrix}\right)}
	\newenvironment{bsmallmatrix}
	{\left[\begin{smallmatrix}}
	{\end{smallmatrix}\right]}
	\setcounter{section}{-1}
\begin{document}
    \maketitle
    \tableofcontents
    \newpage
    \section{Introduction}
        Blahblahblah hoaidhfaiudsf.    

        The main reference of this note is Silverman's \textit{Arithmetic in Elliptic Curves}, and sometimes with examples from \textit{Rational Points on Elliptic Curves} by Silverman \& Tate. Other references: Hartshorne, Vakil, Fulton.
    \section{Basic Constructions}
	\subsection{Curves and rational points}\label{sec-rat}
        We require all varieties in this report to be irreducible.
        \begin{definition}\label{def-curve}
            A \textit{curve} is a projective variety of dimension $1$.
        \end{definition}
        \noindent Therefore, we are only interested in projective varieties and most of the objects will be constructed in them. In order to study the rational solutions of a certain type of equations (that is, on some curves), it is important to base all geometric constructions on some algebraically closed extension of the field we are interested in. Say $k$ is a perfect field and $K$ an algebraic closure of $k$, we denote by $\Gal(K/k)$ the Galois group of $K/k$ (the fixed field of this group is $k$ as $K$ is an algebraic, separable and normal extension of $k$). We can define a natural group action of $\Gal(K/k)$ on the projective space $\mathbb P^n$ over $K$ by $[x_0:\cdots: x_n]^\sigma =[x_0^\sigma:\cdots: x_n^\sigma]$ for any $\sigma$ in the Galois group. We introduce the following notations/definitions.
        \begin{definition}
            The \textit{set of $k$-rational points} $\mathbb P^n(k)$ is the set $\{[x_0:\cdots:x_n]: \forall i, x_i\in K\}$. Given a projective variety $V$, the \textit{$k$-rational points of $V$} is the set $V(k)=V\cap \mathbb P^n(k)$.
        \end{definition}
        \begin{definition}
            A projective variety $V$ is said to be \textit{defined over $k$}, written as $V/k$, if $\mathbb I(V)$ can be generated by homogeneous polynomials with rational coefficients (i.e., in $k[x_0,\dots, x_n]$).
        \end{definition}
        \begin{remark}
            We can also use Galois-theoretic languages:
            \[\mathbb P^n(k)=\{P\in \mathbb P^n:P^\s = P,\forall\s \in\Gal(K/k)\},\]
            and
            \[V(k)=\{P\in V:P^\s = P,\forall\s \in\Gal(K/k)\}\]
            as the fixed field of the Galois group is $k$.
        \end{remark}
        Since we are working with low dimensional cases, it is convenient to define the function field of a projective variety using an affine chart.
        \begin{definition}
            Given a projective variety $V/k$, choose some $U_i=\{x_i\neq 0\}\subseteq \mathbb P^n$. Define the \textit{function field of $V$} to be the quotient field of
            \[k(V)=\frac{k[V\cap U_i]}{\mathbb I(V\cap U_i)\cap k[V\cap U_i]}\]
            and the \textit{larger function field of $V$} to be the quotient field of $K[V\cap U_i]$.
        \end{definition}
        \begin{remark}
            The choice of the chart does not matter --- the resulting fields are isomorphic. The elements in $K(V)$ are of the form $f/g$ where $f, g$ are homogeneous polynomials of the same degree in $n+1$ variables, $g\notin\mathbb I(V)$.
        \end{remark}
        For some projective variety $V$, we can define a group action of $\Gal(K/k)$ on $K[V]$ sending $f=\sum a_{I}x^I$ to $f^\s=\sum \s(a_I)x^I$, that is, acting on the coefficients of $f$. We can further extend this group action to $K(V)$ by defining $(f/g)^\s=f^\s/g^\s$. It is easy to check that the invariant sets in $K[V]$ and $K(V)$ are $k[V]$ and $k(V)$ respectively.
        \begin{remark}\label{remark-gal-on-rational}
            Since $\Gal(K/k)$ consists of field endomorphisms, we have $(f(P))^\s=f^\s(P^\s)$ for all $f$ and points $P$.
        \end{remark}
        
        Suppose $V_1, V_2$ are two projective varieties. For any rational map $\varphi=[f_0:\cdots:f_n]:V_1\to V_2$ where $f_i\in K(V_1)$, we define a group action $\varphi^\s=[f_0^\s:\cdots:f_n^\s]$. And by \autoref{remark-gal-on-rational}, we get $(\varphi(P))^\s=\varphi^\s(P^\s)$.
        \begin{definition}
            A rational map $\varphi=[f_0:\cdots:f_n]$ on $V$ is \textit{defined over $k$} if there exists some nonzero $\lambda\in K$ such that $\lambda f_i\in k(V)$ for all $i$. Equivalently, $\varphi^\s=\varphi$ for all $\s\in\Gal(K/k)$.
        \end{definition}
        \begin{definition}
            For any projective variety $V$ and a smooth point $P\in V$, the maximal ideal at $P$ is $\mathfrak m_P=\{f\in K[V]:f(P)=0\}$. The \textit{local ring of $V$ at $P$}, denoted by $K[V]_P$ is the localization $K[V]_{\mathfrak m_P}$ which is simply the local ring
            \[K[V]_P=\{f/g:f, g\in K[V]\text{ and } g(P)\neq 0\}\]
            A rational function $f\in K(V)$ is said to be regular at $P$ if $f\in K[V]_P$.
        \end{definition}
        \noindent These definitions are equivalent to the mostly seen ones, but it would require some nontrivial arguments to prove the equivalence. See \textcolor{red}{Hartshorne: insert reference here} for instance. They also allow us to do most of our explicit computations on affine coordinate rings, while using the projective varieties to make certain huge theorems work.
        \begin{remark}
            Since for a projective variety $V$ the closure of $V\cap \A^n$ in $\P^n$ is simply $V$, we will always use the affine variety to represent $V$ and other objects on $V$.
        \end{remark}
    \subsection{Weil divisors and principal divisors}
        We reproduce \autoref{def-curve} below.
        \begin{definition}
            A \textit{curve} is a projective variety of dimension $1$.
        \end{definition}
        In our report, the term curve refers to \textbf{smooth} curves, which are curves with all points smooth. Standard algebraic geometry results show that if $P$ is a smooth point of a projective variety $V$, then the dimension (over $K$) of $\mathfrak m_P/\mathfrak m_P^2$ (\textcolor{red}{insert reference here; can be Hartshorne}) equals the dimension of $V$. Thus, given any curve $C$, we have $\dim_K \mathfrak m_P/\mathfrak m_P^2=1$, which gives us the following commutative algebra result:
        \begin{lemma}\label{lemma-ord}
            For all points $P$ in the curve $C$, the local ring $K[C]_P$ is a discrete valuation ring with a valuation defined by
            \[\ord_P(f)=\sup\{d\in\N:f\in\mathfrak m_P^d\}\]
            which is a map $K[C]_P\to\N\cup\{\infty\}$. Furthermore, each valuation $\ord_P$ can be extended to $K(C)$ as $\ord_P(f/g)=\ord_P(f/1)-\ord_P(g/1)$. This map is called \textup{the order of $f/g$ at $P$}.
        \end{lemma}
        \begin{proof}
            \textcolor{red}{Insert reference here}. Possible choice: A-M Prop. 9.2.
        \end{proof}
        \begin{remark}
            The fraction field of a localization of some ring is isomorphic to the fraction field of the original ring. Therefore, $K(C)$ is the fraction field of $K[C]_P$ for any $P$.
        \end{remark}
        \begin{definition}
            The \textit{uniformizer for $C$ at $P$} is some element $t\in K(C)$ such that $\ord_P(t)=1$.
        \end{definition}
        \begin{remark}
            There certainly exists a uniformizer: since every discrete valuation ring is a principal ideal domain, $\mathfrak m_P=(t)$ for some $t$, which must be of order one as suggested by commutative algebra results.

            If $t\in K(C)$ is of order one, then for any $g\in\mathfrak m_P$ (meaning $\ord_P(g)\geqslant 1$), we have
            \[\ord_P(g/t)=\ord_P (g)-1\geqslant 0\]
            so $g/t\in K[C]_P$, meaning there exists some $h\in K[C]_P$ such that $g=th$ and thus $\mathfrak m_P=(t)$.
        \end{remark}
        \noindent To assist further understandings of this valuation, we present the following example.
        \begin{example}
            Consider the curve $y^2=x^3+2x$ over a field $K$ with characteristic not equal to $2$, which is smooth. Let $P=(0, 0)$ then $\mathfrak m_P=(x, y)$ and $\mathfrak m_P^2=(x^2,xy,y^2)$. Since $x=\frac{1}{2}(y^2-x^3)\in\mathfrak m_P^2$ (this also tells us $\ord_P(x)\geqslant 2$), $\mathfrak m_P=(x, y)=(y)$ (in the local ring at $P$). Thus, $\ord_P(y)=1$. Now $y^2=(x^2+2)x$ where $x^2+2\neq 0$ at $P$ which means it's a unit in the local ring, $x=(x^2+2)^{-1}y^2$ and thus
            \[\ord_P(x)=\ord_P(\text{some unit})+2\ord_P(y)=2\]
            where the order of unit must be zero by definition (otherwise $\mathfrak m_P$ wouldn't be maximal). Finally for $2y^2-3x=2x^3+x=(2x^2+1)x$, we have
            \[\ord_P(2y^2-3x)=\ord_P(\text{some unit})+\ord_P(x)=2\]
            where the equality holds since $2x^2+1\neq 0$ at $P=(0, 0)$, i.e., $2x^2+1$ is a unit.
        \end{example}
        
        The order $\ord_P$ at any point $P$ satisfies a very important property, which is an essential part of our algebraic geometry machine.
        \begin{theorem}\label{theorem-fin-ord}
            Let $C$ be a curve and $f\in K(C)$. Then the set $\{P\in C:\ord_P(f)\neq 0\}$ is finite.
        \end{theorem}
        \begin{proof}
            The proof of this theorem by elementary variety theory is rather long and complicated. A short proof using schemes is available in Hartshorne, II.6.1 (\textcolor{red}{insert reference here}).
        \end{proof}
        
\end{document}